\documentclass[11pt]{beamer}
\usepackage{listings} % Include the listings-package
\usepackage[T1]{fontenc}
\usepackage[utf8]{inputenc}
\usepackage[english]{babel}
\usepackage{amsmath}
\usepackage{amssymb, amsfonts, latexsym, cancel}
\usepackage{float}
\usepackage{graphicx}
\usepackage{epstopdf}
\usepackage{subfigure}
\usepackage{hyperref}
%\usepackage{authblk}
\usepackage{blindtext}
\usepackage{booktabs} % Allows the use of \toprule, 
\usepackage{filecontents}
\usepackage{courier} %% Sets font for listing as Courier.
\usepackage{listings}
%\usepackage{listings, xcolor}
\lstset{
tabsize = 2, %% set tab space width
showstringspaces = false, %% prevent space marking in strings, string is defined as the text that is generally printed directly to the console
numbers = left, %% display line numbers on the left
commentstyle = \color{green}, %% set comment color
keywordstyle = \color{blue}, %% set keyword color
stringstyle = \color{red}, %% set string color
rulecolor = \color{black}, %% set frame color to avoid being affected by text color
basicstyle = \small \ttfamily , %% set listing font and size
breaklines = true, %% enable line breaking
numberstyle = \tiny,
}
\usepackage{caption}
\DeclareCaptionFont{white}{\color{white}}
\DeclareCaptionFormat{listing}{\colorbox{gray}{\parbox{\textwidth}{#1#2#3}}}
\captionsetup[lstlisting]{format=listing,labelfont=white,textfont=white}
\definecolor{urlColor}{rgb}{0.06, 0.3, 0.57}
\definecolor{linkColor}{rgb}{0.57, 0.0, 0.04}
\definecolor{fileColor}{rgb}{0.0, 0.26, 0.26}
\hypersetup{
    colorlinks=true,
    linkcolor=linkColor,
    filecolor=fileColor,      
    urlcolor=urlColor,
}
\urlstyle{same}
\setbeamercovered{transparent}
%\usetheme{Boadilla}
\usetheme{CambridgeUS}
%\usetheme{Berkeley}
%\usetheme{Warsaw}
%\usetheme{Madrid}

\title[Introducción]{\bf\Huge David Cheriton}
\subtitle{Interacción Humano Computador}

\author[grupoIHC]
{
	Alex Chaco Huamani \inst{1}
	Renato Eduardo Delgado Huacallo \inst{2}\\
	Christian Wilfredo Ilachoque Hanccoccallo \inst{3}
	Luis Fernando quispe puma \inst{4}
}
\institute[UNSA]
{
\inst{1}% 
Escuela Profesional de Ingenieria de Sistemas\\
Facultad de Producción y Servicio\\
Universidad Nacional San Agustín de Arequipa
}

\date[2020-09-09]{\scriptsize{2020-09-14}}
%\logo{\includegraphics[width=3.0cm]{logo_unsa.jpg}}
\titlegraphic{\includegraphics[width=1.0cm]{logo_unsa.jpg}}

\begin{document}

\begin{frame}
\titlepage
\end{frame}

\begin{frame}

\frametitle{Content}
\tableofcontents
\end{frame}

\section{¿Quién es David Cheriton?}
\begin{frame}
\frametitle{¿Quién es David Cheriton?}
\centering
\includegraphics[width=0.2\textwidth]{cheriton.jpg}
\begin{itemize}
\item Es un informático canadiense, matemático, empresario multimillonario, filántropo y capitalista de riesgo. Es profesor de informática en la Universidad de Stanford , [1] donde fundó y dirige el Grupo de Sistemas Distribuidos.
\item Es un experto en computación distribuida y redes de computadoras.


\end{itemize}
\end{frame}

\section{Diseño De Interfaz Hombre-máquina Para Sistemas De Tiempo Compartido}
\begin{frame}
\frametitle{Diseño De Interfaz Hombre-máquina Para Sistemas De Tiempo Compartido}
\begin{itemize}

\item David Cheriton propuso directrices de diseño de interfaz de usuario para los primeros sistemas informáticos interactivos (tiempo compartido).

\end{itemize}
\end{frame}

\section{Sistemas de Tiempo Compartido}
\begin{frame}
\frametitle{Sistemas de Tiempo Compartido}
\begin{itemize}
\item  Al sistema de tiempo compartido se refiere a un sistema informático que puede utilizarse en línea(conversacionalmente), y que proporciona la base, herramientas para la construcción, mantenimiento y uso de programa de propósito especial.

\end{itemize}
\end{frame}


\section{Propiedades Para El Entorno De Interfaz}
\begin{frame}
\frametitle{Propiedades Para El Entorno De Interfaz}
\begin{itemize}
\item Simple: Proyectar una imagen "virtual" "natural" y sin complicaciones del sistema 
\item Sensible: Responder de forma rápida y significativa a los comandos del usuario.
\item Controlado por el usuario: todas las acciones son iniciadas y controladas por el usuario.
\item Flexible: flexibilidad en la estructura de mando y tolerancia a errores.
\item Estable: capaz de detectar las dificultades del usuario y ayudarlo a volver al diálogo correcto; nunca "interrumpe" al usuario

\end{itemize}
\end{frame}

\begin{frame}
\frametitle{Propiedades Para El Entorno De Interfaz}
\begin{itemize}
\item Protector: proteger al usuario de errores o accidentes costosos, por ejemplo, sobrescribir un archivo.
\item Autodocumentación: Los comandos y las respuestas del sistema son autoexplicativos y la documentación, las explicaciones o el material tutorial forman parte del entorno.
\item Confiable: no conduce a errores no detectados en la comunicación hombre-computadora.
\item Modificable por el usuario: los usuarios sofisticados pueden personalizar su entorno.

\end{itemize}
\end{frame}



\section{Referencias}
\begin{frame}
\frametitle{Referencias}
\begin{itemize}
\item Cheriton, D. R. (1976). Man–machine interface design for time-sharing systems. Proceedings of the ACM National Conference, 362–380.
\end{itemize}
\end{frame}


\end{document}