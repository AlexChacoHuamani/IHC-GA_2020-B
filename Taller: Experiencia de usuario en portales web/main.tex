\documentclass[11pt]{beamer}
\usepackage{listings} % Include the listings-package
\usepackage[T1]{fontenc}
\usepackage[utf8]{inputenc}
\usepackage[english]{babel}
\usepackage{amsmath}
\usepackage{amssymb, amsfonts, latexsym, cancel}
\usepackage{float}
\usepackage{graphicx}
\usepackage{epstopdf}
\usepackage{subfigure}
\usepackage{hyperref}
%\usepackage{authblk}
\usepackage{blindtext}
\usepackage{booktabs} % Allows the use of \toprule, 
\usepackage{filecontents}
\usepackage{courier} %% Sets font for listing as Courier.
\usepackage{listings}
%\usepackage{listings, xcolor}
\lstset{
tabsize = 2, %% set tab space width
showstringspaces = false, %% prevent space marking in strings, string is defined as the text that is generally printed directly to the console
numbers = left, %% display line numbers on the left
commentstyle = \color{green}, %% set comment color
keywordstyle = \color{blue}, %% set keyword color
stringstyle = \color{red}, %% set string color
rulecolor = \color{black}, %% set frame color to avoid being affected by text color
basicstyle = \small \ttfamily , %% set listing font and size
breaklines = true, %% enable line breaking
numberstyle = \tiny,
}
\usepackage{caption}
\DeclareCaptionFont{white}{\color{white}}
\DeclareCaptionFormat{listing}{\colorbox{gray}{\parbox{\textwidth}{#1#2#3}}}
\captionsetup[lstlisting]{format=listing,labelfont=white,textfont=white}
\definecolor{urlColor}{rgb}{0.06, 0.3, 0.57}
\definecolor{linkColor}{rgb}{0.57, 0.0, 0.04}
\definecolor{fileColor}{rgb}{0.0, 0.26, 0.26}
\hypersetup{
    colorlinks=true,
    linkcolor=linkColor,
    filecolor=fileColor,      
    urlcolor=urlColor,
}
\urlstyle{same}
\setbeamercovered{transparent}
%\usetheme{Boadilla}
\usetheme{CambridgeUS}
%\usetheme{Berkeley}
%\usetheme{Warsaw}
%\usetheme{Madrid}

\title[]{\bf\Huge Taller: Experiencia de usuario en portales web}
\subtitle{Interacción Humano Computador}

\author[grupoIHC]
{
	Alex Chaco Huamani \inst{1}
	Renato Eduardo Delgado Huacallo \inst{2}
	Christian Wilfredo Ilachoque Hanccoccallo \inst{3}
	Luis Fernando quispe puma \inst{4}
}
\institute[UNSA]
{
\inst{1}% 
Escuela Profesional de Ingenieria de Sistemas\\
Facultad de Producción y Servicio\\
Universidad Nacional San Agustín de Arequipa
}

\date[2020-10-06]{\scriptsize{2020-10-06}}
%\logo{\includegraphics[width=3.0cm]{logo_unsa.jpg}}
\titlegraphic{\includegraphics[width=1.0cm]{logo_unsa.jpg}}

\begin{document}

\begin{frame}
\titlepage
\end{frame}

\begin{frame}

\frametitle{Contenido}
\tableofcontents
\end{frame}

\section{Introducción}

\begin{frame}
\frametitle{Introducción}
\centering
\includegraphics[width=1\textwidth]{w_g_1.png}
\end{frame}

\begin{frame}
\frametitle{Introducción}
\centering
\includegraphics[width=1\textwidth]{w_g_2.png}
\end{frame}

\begin{frame}
\frametitle{Introducción}
\centering
 Apple es una de las empresas que sigue el principio "La forma sigue la función".
\includegraphics[width=1\textwidth]{apple.png}
\end{frame}

\begin{frame}
\frametitle{Introducción}
\centering
\includegraphics[width=0.5\textwidth]{user_rey.png}\\
{\LARGE USUARIO}
\end{frame}

\section{ACCESIBILIDAD}
\begin{frame}
\frametitle{ACCESIBILIDAD-Información Oculta}
\centering
\includegraphics[width=1\textwidth]{oculta.png}
\end{frame}

\begin{frame}
\frametitle{ACCESIBILIDAD-Popups Escudos de información}
\centering
\includegraphics[width=1\textwidth]{popup.png}
\end{frame}

\begin{frame}
\frametitle{ACCESIBILIDAD-Navegación Oculta}
\centering
\includegraphics[width=1\textwidth]{nav.png}
\end{frame}

\begin{frame}
\frametitle{ACCESIBILIDAD-Navegación Oculta}
\centering
\includegraphics[width=1\textwidth]{nav2.png}
\end{frame}

\begin{frame}
\frametitle{ACCESIBILIDAD-Heros Rotativos}
\centering
\includegraphics[width=1\textwidth]{rotativos.png}
\end{frame}

\begin{frame}
\frametitle{ACCESIBILIDAD-Falta de Contraste}
\centering
\includegraphics[width=1\textwidth]{contraste.png}
\end{frame}

\begin{frame}
\frametitle{ACCESIBILIDAD-Falta de Contraste}
\centering
\includegraphics[width=1\textwidth]{rgb.png}
\end{frame}

\begin{frame}
\frametitle{ACCESIBILIDAD-Falta de Proposito}
\begin{itemize}
\item Los dueños, por supuesto, también tienen objetivos al crear un sitio web. O al menos, deberían de tenerlos, y muy claros.
\end{itemize}
\end{frame}

\section{NAVEGACIÓN}

\begin{frame}{NAVEGACIÓN-Sobrecarga de Menús}
\includegraphics[width=10cm]{UiB-images/1.png}
\end{frame}
\begin{frame}{NAVEGACIÓN-Mega Menú}
\includegraphics[width=10cm]{UiB-images/2.png}
\end{frame}
\begin{frame}{NAVEGACIÓN-Breadcrumb}
\includegraphics[width=10cm]{UiB-images/3.jpg}
\end{frame}
\begin{frame}{NAVEGACIÓN-The Hamburguer Menú}
\textbf{Tab Bar}        \textbf{ ---------------------------------- } \textbf{Menus Progresivos Colapsados}
\includegraphics[width=5cm]{UiB-images/4.jpg}\textbf{-----------}\includegraphics[width=5cm]{UiB-images/5.jpg}
\end{frame}

\section{CONTENIDOS}
\begin{frame}
\frametitle{CONTENIDOS - Mito: Las Personas No Scrollean}
\includegraphics[width=12.0cm]{img/scroll1.PNG}
\includegraphics[width=12.0cm]{img/scroll2.PNG}
\includegraphics[width=12.0cm]{img/scroll3.PNG}

\end{frame}
\begin{frame}
\frametitle{CONTENIDOS - Mito: Las Personas No Scrollean}
\centering Método Acordeon
\begin{figure}
    \includegraphics[width=0.3\textwidth]{img/acordeon1.PNG}
\end{figure}

\end{frame}

\begin{frame}
\frametitle{CONTENIDOS - Above The Fold}
\begin{figure}
    \includegraphics[width=0.6\textwidth]{img/above-the-fold.png}
\end{figure}

\end{frame}

\begin{frame}
\frametitle{CONTENIDOS - Contenido en Blanco}
\begin{figure}
    \includegraphics[width=0.9\textwidth]{img/Espacio-en-blanco.PNG}
\end{figure}

\end{frame}
\begin{frame}
\frametitle{CONTENIDOS - Contenido en Blanco MACRO}
\begin{figure}
    \includegraphics[width=0.65\textwidth]{img/Espacio-en-blanco-MACRO.PNG}
\end{figure}
\end{frame}

\begin{frame}
\frametitle{CONTENIDOS - Contenido en Blanco Micro}
\begin{figure}
    \includegraphics[width=0.9\textwidth]{img/Espacio-en-blanco-micro.PNG}
\end{figure}
\end{frame}


\section{TEXTOS}
\begin{frame}
\frametitle{TEXTOS - Jerarquías}
\centering
\begin{itemize}
\item Siempre hay que tener en cuenta las jerarquías del tamaño del texto (título, subtítulo, etc.)\\
\centering
\includegraphics[width=0.4\textwidth]{img1.jpg}

\end{itemize}
\end{frame}



\begin{frame}
\frametitle{TEXTOS - Alineación justificada vs Alineación a la izquierda}
\begin{itemize}

\item Por un lado la alineación justificada hace ver el contenido mucho más ordenado.
\item Por otro lado la alineación justificada hace que a veces exista un mayor espacio entre las palabras y puede causar cansancio visual.\\
\centering
\includegraphics[width=0.7\textwidth]{img2.jpg}

\end{itemize}
\end{frame}


\begin{frame}
\frametitle{TEXTOS - Ancho de lineal ideal}
\begin{itemize}
\item  Es recomendable usar enrte 60 a 70 carcteres por línea.\\
\centering
\includegraphics[width=0.7\textwidth]{img3.jpg}

\end{itemize}
\end{frame}


\begin{frame}
\frametitle{TEXTOS - Mayúsculas y Minúsculas}
\begin{itemize}
\item Hay que tener en cuenta que usamos las mayúsculas lo que se quiere resaltar.\\
\centering
\includegraphics[width=0.6\textwidth]{img4.jpg}

\end{itemize}
\end{frame}


\section{Referencias}
\begin{frame}
\frametitle{Referencias}
\begin{itemize}
\item HIC 2020 [HCI 2020]. (2020,09,20). Taller: Experiencia de usuario en portales web - HCI 2020 [Archivo de video]. Recuperado de https://www.youtube.com/watch?v=nNx2T-yU0NI
\item Hassan Montero, Y. (2002). Cómo leen los usuarios en la Web. No sólo usabilidad, (1).
\end{itemize}
\end{frame}


\end{document}